\documentclass{article}
\usepackage{graphicx}
\usepackage{amsmath}
\usepackage{array}
\usepackage{fancyhdr}
\usepackage{amssymb}
\usepackage[shortlabels]{enumitem}

\DeclareMathOperator{\R}{\mathbb R}

\pagestyle{fancy}
\fancyhead[L]{Banghao Chi}
\fancyhead[C]{Homework 2}
\fancyhead[R]{14th Feb}

\fancyfoot[C]{\thepage}

\renewcommand{\headrulewidth}{0.5pt}
\renewcommand{\footrulewidth}{0.5pt}

\begin{document}

\section*{Exercise 1}
Solve the following LP using the two phase simplex algorithm:
\begin{align*}
\text{maximize} \quad & 3y + z \\
x, y, z \in \mathbb{R} & \\
\text{subject to} \quad & x + 2y + z \leq 2, \\
& 2x + y - z \leq -1, \\
& 3x + 2y + z \leq 3, \\
& x, y, z \geq 0
\end{align*}

\textbf{Solution:} \\

Since I registered the three-credits section, I'll skip this exercise.

\newpage

\section*{Exercise 2}
Use lexicographic pivoting to solve the following linear program:
\begin{align*}
\text{maximize} \quad & x - y \\
x, y \in \mathbb{R} & \\
\text{subject to} \quad & x - 2y \leq 0, \\
& x - 3y \leq 0, \\
& y \leq 3, \\
& x, y \geq 0
\end{align*}

\textbf{Solution:}

\newpage

\section*{Exercise 3}
Consider the following LP:
\begin{align*}
\text{maximize} \quad & c^T x \\
x \in \mathbb{R}^n & \\
\text{subject to} \quad & Ax \leq 0, \\
& x \geq 0
\end{align*}
Show that either $x = 0$ is an optimal solution or else the LP is unbounded. \\

\textbf{Solution:} \\

We first note that $x = 0$ is always a feasible solution. \\

If $x^*$ is a feasible solution other than 0, then:
\begin{align*}
Ax^* &\leq 0 \\
x^* &\geq 0
\end{align*}

Consider any scalar $\lambda \geq 0$ and plugging $\lambda x^*$ into the LP, we get:
\begin{align*}
A(\lambda x^*) &= \lambda(Ax^*) \leq 0 \quad \text{(since $Ax^* \leq 0$ and $\lambda \geq 0$)} \\
\lambda x^* &\geq 0 \quad \text{(since $x^* \geq 0$ and $\lambda \geq 0$)}
\end{align*}

which shows that if $x^*$ is feasible, then $\lambda x^*$ is also feasible for any $\lambda \geq 0$. \\

Plugging this into the objective function, we get:
\begin{align*}
c^T(\lambda x^*) = \lambda(c^T x^*)
\end{align*}

\begin{itemize}
\item If $c^T x^* > 0$, then as $\lambda \to \infty$, the objective value $\lambda(c^T x^*)$ will approach infinity, making the problem unbounded.
\item If $c^T x^* < 0$, then $x^*$ cannot be optimal since $x = 0$ would give a better objective value which is 0, which in this case makes $x = 0$ the optimal solution since $x^*$ represents any feasible solution except for $x = 0$.
\item If $c^T x^* = 0$, then $x = 0$ is also optimal.
\end{itemize}

Therefore, either:
\begin{itemize}
\item $x = 0$ is optimal (when no feasible solution exists with positive objective value), or
\item The problem is unbounded (when there exists a feasible solution $x^*$ with $c^T x^* > 0$)
\end{itemize}

\newpage

\section*{Exercise 4}
Prove that the variable that becomes nonbasic in one iteration of the simplex method cannot become basic in the next iteration. \\

\textbf{Solution:} \\

Without loss of generality, consider a maximization problem. \\

1) Suppose variable $x_k$ is basic in iteration $t$ and becomes nonbasic, then becomes basic again in iteration $t+1$. \\

2) At iteration $t$:
    \begin{itemize}
    \item Let $B_t$ be the set of basic variables
    \item Let $N_t$ be the set of nonbasic variables
    \item $x_k \in B_t$ initially
    \item Let $x_e$ be the entering variable, where $e \in N_t$
    \end{itemize}

3) For $x_k$ to leave the basis:
    \begin{itemize}
    \item ${c}_e > 0$ (since it's a maximization problem)
    \item $a_{ke} > 0$ (pivot element must be positive)
    \item $c_k = 0$ (since $x_k$ is nonbasic)
    \item $x_k$ gives the minimum ratio.
    \end{itemize}

\[
\begin{array}{c|cc|c}
    & x_e & x_k & \\
\hline
    x_k & a_{ke} & 1 & \\
    \hline
    & c_e & c_k = 0 &  \\
\end{array}
\]

4) After pivot operation (beginning of iteration $t+1$):
    \begin{itemize}
    \item The new reduced cost for $x_k$ is:
     ${c}_k^{new} = 0 - \frac{{c}_e}{a_{ke}} \cdot 1 = - \frac{{c}_e}{a_{ke}} < 0$
    \end{itemize}

5) For $x_k$ to enter in iteration $t+1$:
    \begin{itemize}
    \item It would need ${c}_k^{new} > 0$ (since it's a maximization problem)
    \item But we proved ${c}_k^{new} < 0$
    \end{itemize}

Contradiction. Therefore, a variable that becomes nonbasic in iteration $t$ cannot become basic in iteration $t+1$.

\newpage

\section*{Exercise 5}
Let $A, B \subseteq \mathbb{R}^n$ be nonempty convex sets. Determine with explanation the following:
\begin{enumerate}
\item[(a)] Can $A \cup B$ be convex?
\item[(b)] Is $A \cap B$ always convex?
\item[(c)] Can $B \setminus A$ be convex?
\item[(d)] Can $\mathbb{R}^n \setminus A$ be convex?
\end{enumerate}

\textbf{Solution:} \\

(a) Can $A \cup B$ be convex?
\begin{itemize}
\item $A \cup B$ can be convex, but it can also be non-convex.
    \begin{enumerate}
    \item Convex example: Let $A = [0,1]$ and $B = [1,2]$ in $\mathbb{R}$. Then $A \cup B = [0,2]$, which is convex.
    \item Non-convex example: Let $A = [0,1]$ and $B = [2,3]$ in $\mathbb{R}$. Then $A \cup B = [0,1] \cup [2,3]$ is not convex because if we take $x = 0.5 \in A$ and $y = 2.5 \in B$, then their midpoint $\frac{x+y}{2} = 1.5 \notin A \cup B$.
    \end{enumerate}
\end{itemize}

(b) Is $A \cap B$ always convex?
\begin{itemize}
\item Yes, the intersection of any number of convex sets is always convex.
\item Proof: Let $x, y \in A \cap B$ and $\lambda \in [0,1]$. Then:
    \begin{enumerate}
    \item Since $x,y \in A$ and $A$ is convex, $\lambda x + (1-\lambda)y \in A$
    \item Since $x,y \in B$ and $B$ is convex, $\lambda x + (1-\lambda)y \in B$
    \item Therefore, $\lambda x + (1-\lambda)y \in A \cap B$
    \end{enumerate}
\end{itemize}

(c) Can $B \setminus A$ be convex?
\begin{itemize}
\item $B \setminus A$ can be convex, but it can also be non-convex.
\item Convex example: In $\mathbb{R}$, let $B = [0,2]$ and $A = [2,3]$. Then $B \setminus A = [0,2)$ is convex.
\item Non-convex example: In $\mathbb{R}$, let $B = [0,3]$ and $A = [1,2]$. Then $B \setminus A = [0,1) \cup (2,3]$ is not convex.
\end{itemize}

(d) Can $\mathbb{R}^n \setminus A$ be convex?
\begin{itemize}
\item $\mathbb{R}^n \setminus A$ can be convex, but it can also be non-convex.
\item Convex example: Let $A = \{x \in \mathbb{R}^n : \|x\| \geq 1\}$. Then $\mathbb{R}^n \setminus A = \{x \in \mathbb{R}^n : \|x\| < 1\}$(the unit ball) is convex.
\item Non-convex example: Let $A = \{x \in \mathbb{R}^n : \|x\| < 1\}$. Then $\mathbb{R}^n \setminus A = \{x \in \mathbb{R}^n : \|x\| \geq 1\}$ which is not convex(for any two points in $\mathbb{R}^n$ symmetric about the origin, their midpoint, which is the origin itself, is not in $\mathbb{R}^n \setminus A$), so $\mathbb{R}^n \setminus A$ is not convex.
\end{itemize}

\end{document}