\documentclass{article}
\usepackage{graphicx}
\usepackage{amsmath}
\usepackage{array}
\usepackage{fancyhdr}
\usepackage{amssymb}
\usepackage[shortlabels]{enumitem}

\DeclareMathOperator{\R}{\mathbb R}

\pagestyle{fancy}
\fancyhead[L]{Banghao Chi}
\fancyhead[C]{Homework 2}
\fancyhead[R]{14th Feb}

\fancyfoot[C]{\thepage}

\renewcommand{\headrulewidth}{0.5pt}
\renewcommand{\footrulewidth}{0.5pt}

\begin{document}

\section*{Exercise 1}
Solve the following LP using the two phase simplex algorithm:
\begin{align*}
\text{maximize} \quad & 3y + z \\
x, y, z \in \mathbb{R} & \\
\text{subject to} \quad & x + 2y + z \leq 2, \\
& 2x + y - z \leq -1, \\
& 3x + 2y + z \leq 3, \\
& x, y, z \geq 0
\end{align*}

\textbf{Solution:} \\

\newpage

\section*{Exercise 2}
Use lexicographic pivoting to solve the following linear program:
\begin{align*}
\text{maximize} \quad & x - y \\
x, y \in \mathbb{R} & \\
\text{subject to} \quad & x - 2y \leq 0, \\
& x - 3y \leq 0, \\
& y \leq 3, \\
& x, y \geq 0
\end{align*}

\textbf{Solution:}

\newpage

\section*{Exercise 3}
Consider the following LP:
\begin{align*}
\text{maximize} \quad & c^T x \\
x \in \mathbb{R}^n & \\
\text{subject to} \quad & Ax \leq 0, \\
& x \geq 0
\end{align*}
Show that either $x = 0$ is an optimal solution or else the LP is unbounded. \\

\textbf{Solution:}

\newpage

\section*{Exercise 4}
Prove that the variable that becomes nonbasic in one iteration of the simplex method cannot become basic in the next iteration. \\

\textbf{Solution:}

\newpage

\section*{Exercise 5}
Let $A, B \subseteq \mathbb{R}^n$ be nonempty convex sets. Determine with explanation the following:
\begin{enumerate}
\item[(a)] Can $A \cup B$ be convex?
\item[(b)] Is $A \cap B$ always convex?
\item[(c)] Can $B \setminus A$ be convex?
\item[(d)] Can $\mathbb{R}^n \setminus A$ be convex?
\end{enumerate}

\textbf{Solution:} \\

(a) Can $A \cup B$ be convex?
\begin{itemize}
\item $A \cup B$ can be convex, but it can also be non-convex.
    \begin{enumerate}
    \item Convex example: Let $A = [0,1]$ and $B = [1,2]$ in $\mathbb{R}$. Then $A \cup B = [0,2]$, which is convex.
    \item Non-convex example: Let $A = [0,1]$ and $B = [2,3]$ in $\mathbb{R}$. Then $A \cup B = [0,1] \cup [2,3]$ is not convex because if we take $x = 0.5 \in A$ and $y = 2.5 \in B$, then their midpoint $\frac{x+y}{2} = 1.5 \notin A \cup B$.
    \end{enumerate}
\end{itemize}

(b) Is $A \cap B$ always convex?
\begin{itemize}
\item Yes, the intersection of any number of convex sets is always convex.
\item Proof: Let $x, y \in A \cap B$ and $\lambda \in [0,1]$. Then:
    \begin{enumerate}
    \item Since $x,y \in A$ and $A$ is convex, $\lambda x + (1-\lambda)y \in A$
    \item Since $x,y \in B$ and $B$ is convex, $\lambda x + (1-\lambda)y \in B$
    \item Therefore, $\lambda x + (1-\lambda)y \in A \cap B$
    \end{enumerate}
\end{itemize}

(c) Can $B \setminus A$ be convex?
\begin{itemize}
\item $B \setminus A$ can be convex, but it can also be non-convex.
\item Convex example: In $\mathbb{R}$, let $B = [0,2]$ and $A = [2,3]$. Then $B \setminus A = [0,2)$ is convex.
\item Non-convex example: In $\mathbb{R}$, let $B = [0,3]$ and $A = [1,2]$. Then $B \setminus A = [0,1) \cup (2,3]$ is not convex.
\end{itemize}

(d) Can $\mathbb{R}^n \setminus A$ be convex?
\begin{itemize}
\item $\mathbb{R}^n \setminus A$ can be convex, but it can also be non-convex.
\item Convex example: Let $A = \{x \in \mathbb{R}^n : \|x\| \geq 1\}$. Then $\mathbb{R}^n \setminus A = \{x \in \mathbb{R}^n : \|x\| < 1\}$(the unit ball) is convex.
\item Non-convex example: Let $A = \{x \in \mathbb{R}^n : \|x\| < 1\}$. Then $\mathbb{R}^n \setminus A = \{x \in \mathbb{R}^n : \|x\| \geq 1\}$ which is not convex(for any two points in $\mathbb{R}^n$ symmetric about the origin, their midpoint, which is the origin itself, is not in $\mathbb{R}^n \setminus A$), so $\mathbb{R}^n \setminus A$ is not convex.
\end{itemize}

\end{document}