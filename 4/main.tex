\documentclass{article}
\usepackage{graphicx}
\usepackage{amsmath}
\usepackage{array}
\usepackage{fancyhdr}
\usepackage{amssymb}
\usepackage[shortlabels]{enumitem}

\DeclareMathOperator{\R}{\mathbb R}

\pagestyle{fancy}
\fancyhead[L]{Banghao Chi}
\fancyhead[C]{Homework 4}
\fancyhead[R]{2nd Mar}

\fancyfoot[C]{\thepage}

\renewcommand{\headrulewidth}{0.5pt}
\renewcommand{\footrulewidth}{0.5pt}

\begin{document}

\section*{Exercise 1}
Use the simplex method to solve the LP below. Then, use your final simplex tableau to find the optimal dual solution.

\begin{align*}
\text{maximize} \quad & x - y + z\\
x, y, z \in \mathbb{R} \quad & \\
\text{subject to} \quad & x + 2y + z \leq 5,\\
& 2x + y + z \leq 6,\\
& x, y, z \geq 0
\end{align*}

\textbf{Solution:} \\

Add slack variables $s_1$ and $s_2$ to that we end up with standard form:
\begin{align*}
x + 2y + z + s_1 &= 5\\
2x + y + z + s_2 &= 6\\
x, y, z, s_1, s_2 &\geq 0
\end{align*}

Initial simplex tableau:
$$
\begin{array}{c|ccccc|c}
 & x & y & z & s_1 & s_2 &  \\
\hline
s_1 & 1 & 2 & 1 & 1 & 0 & 5 \\
s_2 & 2 & 1 & 1 & 0 & 1 & 6 \\
\hline
-f & 1 & -1 & 1 & 0 & 0 & 0
\end{array}
$$

Select $x$ as the entering variable. Ratios: $5/1 = 5 > 6/2 = 3$. So $s_2$ leaves.

$$
\begin{array}{c|ccccc|c}
 & x & y & z & s_1 & s_2 &  \\
\hline
s_1 & 0 & \frac{3}{2} & \frac{1}{2} & 1 & -\frac{1}{2} & 2 \\
x & 1 & \frac{1}{2} & \frac{1}{2} & 0 & \frac{1}{2} & 3 \\
\hline
-f & 0 & -\frac{3}{2} & \frac{1}{2} & 0 & -\frac{1}{2} & -3
\end{array}
$$

Select $z$ as the entering variable. Ratios: $2/(\frac{1}{2}) = 4 < 3/(\frac{1}{2}) = 6$. So $s_1$ leaves.

$$
\begin{array}{c|ccccc|c}
 & x & y & z & s_1 & s_2 &  \\
\hline
z & 0 & 3 & 1 & 2 & -1 & 4 \\
x & 1 & -1 & 0 & -1 & 1 & 1 \\
\hline
-f & 0 & -3 & 0 & -1 & 0 & -5
\end{array}
$$

Since all coefficients in the objective row are non-negative, we've reached optimality.

The optimal solution is:
\begin{align*}
x &= 1\\
y &= 0\\
z &= 4\\
\end{align*}

We know that the optimal dual solution is given by:
\begin{align*}
    u^T = c_\mathcal{B}^T A_\mathcal{B}^{-1}
\end{align*}

Therefore, since we have:
\begin{align*}
c_\mathcal{B}^T = [1, 1]
\end{align*}
\begin{align*}
A_\mathcal{B} = \begin{bmatrix} 
1 & 1 \\ 
2 & 1 
\end{bmatrix}
\end{align*}

Compute the inverse:
\begin{align*}
A_\mathcal{B}^{-1} &= \frac{1}{1 \cdot 1 - 2 \cdot 1} \begin{bmatrix} 
1 & -1 \\ 
-2 & 1 
\end{bmatrix} \\
&= \frac{1}{-1} \begin{bmatrix} 
1 & -1 \\ 
-2 & 1 
\end{bmatrix} \\
&= \begin{bmatrix} 
-1 & 1 \\ 
2 & -1 
\end{bmatrix}
\end{align*}

Therefore, the optimal dual solution is:
\begin{align*}
u^T &= c_\mathcal{B}^T A_\mathcal{B}^{-1} \\
&= [1, 1] \begin{bmatrix} 
-1 & 1 \\ 
2 & -1 
\end{bmatrix} \\
&= [-1 + 2, 1 - 1] \\
&= [1, 0]
\end{align*}

\newpage

\section*{Exercise 2}
Solve the following LP using the dual simplex method:

\begin{align*}
\text{minimize} \quad & x + y\\
x, y \in \mathbb{R} \quad & \\
\text{subject to} \quad & 2x + y \geq 3,\\
& x + 3y \geq 4,\\
& x, y \geq 0
\end{align*}

\textbf{Solution:} \\

Inverse the inequality constraints and add slack variables:
\begin{align*}
-2x - y + s_1 &= -3\\
-x - 3y + s_2 &= -4\\
x, y, s_1, s_2 &\geq 0
\end{align*}

Initial simplex tableau(which is dual feasible):
$$
\begin{array}{c|cccc|c}
 & x & y & s_1 & s_2 &  \\
\hline
s_1 & -2 & -1 & 1 & 0 & -3 \\
s_2 & -1 & -3 & 0 & 1 & -4 \\
\hline
-f & 1 & 1 & 0 & 0 & 0
\end{array}
$$

$s_1$ leaves. Ratio: $1 / |-2| = 1/2 < 1 / |-1| = 1$. So $x$ enters.

$$
\begin{array}{c|cccc|c}
 & x & y & s_1 & s_2 &  \\
\hline
x & 1 & \frac{1}{2} & -\frac{1}{2} & 0 & \frac{3}{2} \\
s_2 & 0 & -\frac{5}{2} & -\frac{1}{2} & 1 & -\frac{5}{2} \\
\hline
-f & 0 & \frac{1}{2} & \frac{1}{2} & 0 & -\frac{3}{2}
\end{array}
$$

$s_2$ leaves. Ratio: $1 / 2 / |-5/2| = 1/5 < 1 / 2 / |- 1 / 2| = 1$. So $y$ enters.

$$
\begin{array}{c|cccc|c}
 & x & y & s_1 & s_2 &  \\
\hline
x & 1 & 0 & -\frac{3}{5} & \frac{1}{5} & 1 \\
y & 0 & 1 & \frac{1}{5} & -\frac{2}{5} & 1 \\
\hline
-f & 0 & 0 & \frac{2}{5} & \frac{1}{5} & -2
\end{array}
$$

Since all coefficients in the objective row are non-negative, we've reached optimality. \\

The optimal solution is:
\begin{align*}
x &= 1\\
y &= 1\\
\end{align*}

The optimal value is 2.

\newpage

\section*{Exercise 3}
Solve the following LP using the dual two-phase simplex method:

\begin{align*}
\text{maximize} \quad & 3x - 2y\\
x, y \in \mathbb{R} \quad & \\
\text{subject to} \quad & -2x + y \geq 1,\\
& -x + y \leq 2,\\
& x, y \geq 0
\end{align*}

\textbf{Solution:} \\

Since I registered 3 credits section, I'd like to skip this problem.

\newpage

\section*{Exercise 4}
Prove that each iteration of the dual simplex method improves the dual optimal value. \\

\textbf{Solution:} \\

We start with a dual feasible but primal infeasible basic solution. \\

At each iteration:
\begin{enumerate}
    \item $x_r$ with $x_r < 0$ (primal infeasible) leaves the basis.
    \item $x_s$ enters the basis using the following criteria:
\begin{align*}
s = \arg\min_{j \in N, a_{rj} < 0} \left\{ \frac{c_j}{|a_{rj}|} \right\}
\end{align*}
which maintains the dual feasibility after the pivot.
\end{enumerate}

The change in the dual objective value after this pivot is:
\begin{align*}
\Delta (-z) = -\frac{c_s \cdot x_r}{a_{rs}}
\end{align*}

Since:
\begin{itemize}
    \item $x_r < 0$ (the leaving variable is primal infeasible)
    \item $a_{rs} < 0$ (from the pivot selection criteria)
\end{itemize}

And:
\begin{itemize}
    \item For maximization: $c_s \geq 0$ due to dual feasibility. Therefore $\Delta (-z) \leq 0$, meaning $\Delta z \geq 0$, objective value improves.
    \item For minimization: $c_s \leq 0$ due to dual feasibility. Therefore $\Delta (-z) \geq 0$, meaning $\Delta z \leq 0$, objective value improves.
\end{itemize}

In both cases, the objective value improves in the desired direction, and thus we complete the proof.

\newpage

\section*{Exercise 5}
Suppose that $x_0$ is a feasible solution to the following LP:

\begin{align*}
\text{minimize} \quad & c^\top x\\
x \in \mathbb{R}^n \quad & \\
\text{subject to} \quad & Ax = b,\\
& x \geq 0
\end{align*}

Suppose additionally that $y \in \mathbb{R}^n$ satisfies $Ay = 0$, $c^\top y < 0$, and $y \geq 0$. Prove that the LP is unbounded. \\

\textbf{Solution:} \\

Let $x_t = x_0 + ty$ for any $t \geq 0$ and we verify that it is feasible. \\

1. Equality constraints:
   \begin{align*}
   Ax_t &= A(x_0 + ty)\\
   &= Ax_0 + tAy\\
   &= b + t \cdot 0\\
   &= b
   \end{align*}

   So $x_t$ satisfies $Ax = b$ for all $t \geq 0$. \\

2. Non-negativity:
   Since $x_0 \geq 0$, $y \geq 0$, and $t \geq 0$, we have $x_t = x_0 + ty \geq 0$. \\

Therefore, $x_t$ is feasible for all $t \geq 0$. \\

For the objective value:
\begin{align*}
c^{\top}x_t &= c^{\top}(x_0 + ty)\\
&= c^{\top}x_0 + tc^{\top}y
\end{align*}

Since $c^{\top}y < 0$, as $t \to \infty$, we have $c^{\top}x_t \to -\infty$. \\

Therefore, the LP is unbounded.

\end{document}