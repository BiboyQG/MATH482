\documentclass{article}
\usepackage{graphicx}
\usepackage{amsmath}
\usepackage{array}
\usepackage{fancyhdr}
\usepackage{amssymb}
\usepackage[shortlabels]{enumitem}

\DeclareMathOperator{\R}{\mathbb R}

\pagestyle{fancy}
\fancyhead[L]{Banghao Chi}
\fancyhead[C]{Homework 5}
\fancyhead[R]{10th Mar}

\fancyfoot[C]{\thepage}

\renewcommand{\headrulewidth}{0.5pt}
\renewcommand{\footrulewidth}{0.5pt}

\begin{document}

\section*{Exercise 1}
Here is an LP and an optimal tableau
\begin{align*}
\text{maximize} \quad & 5x + 4y \\
x, y \in \mathbb{R} & \\
\text{subject to} \quad & x \leq 10, \\
& y \leq 10, \\
& x + 2y \leq 26, \\
& x, y \geq 0
\end{align*}

$$\begin{array}{c|ccccc|c}
& x & y & s_1 & s_2 & s_3 & \\
\hline
x & 1 & 0 & 1 & 0 & 0 & 10 \\
s_2 & 0 & 0 & 1/2 & 1 & -1/2 & 2 \\
y & 0 & 1 & -1/2 & 0 & 1/2 & 8 \\
\hline
-z & 0 & 0 & -3 & 0 & -2 & -82
\end{array}$$

For small values of $\delta$, describe the effect of the following changes on the objective function and on the optimal primal or dual solution. For what range of $\delta$ do these changes hold?

\begin{enumerate}[label=(\alph*)]
    \item The objective function changes from $5x + 4y$ to $(5 + \delta)x + 4y$.

    \item The upper bound on $x$ changes from $x \leq 10$ to $x \leq 10 + \delta$.

    \item The upper bound on $y$ changes from $y \leq 10$ to $y \leq 10 + \delta$.
\end{enumerate}

\textbf{Solution: }

\begin{enumerate}[label=(\alph*)]
    \item For dual feasibility, we need $-(3+\delta) \leq 0$, which gives us $\delta \geq -3$. If $\delta$ changes within this range, the optimal solution remains the same but the objective value changes to $82 + 10\delta$ (since $x = 10$ in the optimal solution).

    \item This changes the RHS of a tight constraint whose slack variable is $s_1$. So we compute the negative ratios $-\frac{\text{entry in RHS column}}{\text{entry in $s_1$'s column}}$ for each row:
    \begin{itemize}
        \item For $x$ row: $-\frac{10}{1} = -10$
        \item For $s_2$ row: $-\frac{2}{1/2} = -4$
        \item For $y$ row: $-\frac{8}{-1/2} = 16$
    \end{itemize}

    So the range is $-4 \leq \delta \leq 16$. For $\delta$ in this range, $x = 10 + \delta$, $y = 8$, and the objective value becomes $82 + 3\delta$.

    \item This changes the RHS of a slack constraint whose slack variable is $s_2$. Since $s_2$ is basic with value 2 (not tight), the only limit on $\delta$ is $\delta \geq -2$. There is no change to the optimal solution or objective value as long as $\delta > -2$.
\end{enumerate}

\newpage

\section*{Exercise 2}
Use the dual simplex method to add the constraint $x+5y \leq 11$ to the LP from problem 1 and find the new optimal solution. \\

\textbf{Solution: } \\

Since I registered for the three credits section, I will not be doing this problem.

\newpage

\section*{Exercise 3}
Alice and Bob each have two coins: a nickel (5 cents) and a dime (10 cents). They simultaneously put a coin down on the table. If the coins are equal in value, Alice wins Bob's coin; if Alice's coin is more valuable, Bob wins Alice's coin; if Bob's coin is more valuable, nothing happens. \\

Determine optimal strategies for Alice and Bob. \\

\textbf{Solution: } \\

According to the question, we have the following payoff matrix showing Alice's payoffs (in cents):

\[
\begin{array}{c|cc}
& \text{Bob: 5} & \text{Bob: 10} \\
\hline
\text{Alice: 5} & 5 & 0 \\
\text{Alice: 10} & -10 & 10 \\
\end{array}
\]

\begin{itemize}
    \item For Alice, we want the find $\max(\min(\text{Rows}))$, and since $0 > -10$, putting a nickel down is the best strategy for Alice, since it presents a better worst case scenario.
    \item For Bob, we want the find $\min(\max(\text{Columns}))$, and since $5 < 10$, putting a nickel down is the best strategy for Bob, since it presents a better worst case scenario.
\end{itemize}

\newpage

\section*{Exercise 4}
Two players simultaneously throw out one or two fingers and call out their guess as to what the total sum of the outstretched fingers will be. If exactly one of the players guesses correctly, the player with the correct guess receives payment equal to the guess. In all other cases, it is a draw. Formulate the row and column players's problems as LPs. \\

You may assume the players behave rationally, i.e., if a player throws out one finger, their guess will be strictly greater than one. \\

\textbf{Solution: } \\

Let $(x,y)$ be a strategy of throwing out $x$ finger(s) and call out $y$ as their guesses. According to the question, we have the following payoff matrix for the row player:

\[
\begin{array}{c|ccccc}
 & (1,2) & (1,3) & (1,4) & (2,3) & (2,4) \\
\hline
(1,2) & 0 & 2 & 2 & -3 & 0 \\
(1,3) & -2 & 0 & 0 & 0 & 3 \\
(1,4) & -2 & 0 & 0 & -3 & 0 \\
(2,3) & 3 & 0 & 3 & 0 & -4 \\
(2,4) & 0 & -3 & 0 & 4 & 0 \\
\end{array}
\] \\

Let $x_i$ be the probability for the row player to choose strategy $i$, then we have the following row player's LP:
\begin{align*}
\text{maximize} & \quad u \\
x \in \mathbb{R}^5, u \in \mathbb{R} & \\
\text{subject to} & \quad u \leq -2x_2 - 2x_3 + 3x_4, \\
& \quad u \leq 2x_1 - 3x_5, \\
& \quad u \leq 2x_1 + 3x_4, \\
& \quad u \leq -3x_1 - 3x_3 + 4x_5, \\
& \quad u \leq 3x_2 - 4x_4, \\
& \quad x_i \geq 0, \quad \forall i \in \{1,2,3,4,5\}, \\
& \quad \sum_{i=1}^5 x_i = 1
\end{align*}

Let $y_i$ be the probability for the column player to choose strategy $i$, then we have the following column player's LP:
\begin{align*}
\text{minimize} & \quad v \\
y \in \mathbb{R}^5, v \in \mathbb{R} & \\
\text{subject to} & \quad v \geq 2y_2 + 2y_3 - 3y_4, \\
& \quad v \geq -2y_1 + 3y_5, \\
& \quad v \geq -2y_1 - 3y_4, \\
& \quad v \geq 3y_1 + 3y_3 - 4y_5, \\
& \quad v \geq -3y_2 + 4y_4, \\
& \quad y_i \geq 0, \quad \forall i \in \{1,2,3,4,5\}, \\
& \quad \sum_{i=1}^5 y_i = 1
\end{align*}

\newpage

\section*{Exercise 5}
Consider a zero-sum game with the payoff matrix below:

$$\begin{array}{c|ccc}
& \text{Bob: } a & \text{Bob: } b & \text{Bob: } c \\
\hline
\text{Alice: } A & -2 & 3 & -4 \\
\text{Alice: } B & 5 & -6 & 7
\end{array}$$

Suppose Alice plays the mixed strategy $\left[2/3 \quad 1/3\right]$. \\

Find the range of values of $p$ such that Bob playing option $b$ with probability $p$ and option $c$ with probability $1 - p$ dominates the pure strategy of option $a$. \\

\textbf{Solution: } \\

Since we have $x$ which is Alice's strategy, we can compute that: $$x^T A = [1/3 \; 0 \; -1/3]$$

So when not choosing option $a$ at all, we want $$x^T Ay \leq 1/3, \quad y = [0, p, 1-p]^T$$

Therefore, we have: $-\frac{1-p}{3} \leq \frac{1}{3} \implies p \leq 2$. \\

Therefore, the range of $p$ is $0 \leq p \leq 1$.

\end{document}