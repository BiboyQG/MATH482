\documentclass{article}
\usepackage{graphicx}
\usepackage{amsmath}
\usepackage{array}
\usepackage{fancyhdr}
\usepackage{amssymb}
\usepackage[shortlabels]{enumitem}

\DeclareMathOperator{\R}{\mathbb R}

\pagestyle{fancy}
\fancyhead[L]{Banghao Chi}
\fancyhead[C]{Homework 3}
\fancyhead[R]{28th Feb}

\fancyfoot[C]{\thepage}

\renewcommand{\headrulewidth}{0.5pt}
\renewcommand{\footrulewidth}{0.5pt}

\begin{document}

\section*{Exercise 1}
Consider the following LP:
\begin{align*}
\max_{x, y, z \in \mathbb{R}} \quad & x + y + z \\
\text{subject to} \quad & 2x + y + 2z \leq 14, \\
& x + z \leq 8, \\
& 2x + 2y - z \leq 18, \\
& x, y, z \geq 0
\end{align*}

Write down the dual LP. \\

\textbf{Solution:} \\

\text{The dual LP is:} \\
\begin{align*}
\min_{u, v, w \in \mathbb{R}} \quad & 14u + 8v + 18w \\
\text{subject to} \quad & 2u + v + 2w \geq 1, \\
& u + 2w \geq 1, \\
& 2u + v - w \geq 1, \\
& u, v, w \geq 0
\end{align*}

\newpage

\section*{Exercise 2}
Solve the dual of the following LP (no need to find an optimal vector, just the optimum value):
\begin{align*}
\min_{x \in \mathbb{R}^4} \quad & x_1 + x_2 + x_3 + x_4 \\
\text{subject to} \quad & x_1 + x_3 = 4, \\
& x_2 + x_3 + x_4 = 4, \\
& x_3 + x_4 = 1, \\
& x \geq 0
\end{align*}

\textbf{Solution:} \\

The dual LP is:
\begin{align*}
\max_{u, v, w \in \mathbb{R}} \quad & 4u + 4v + w \\
\text{subject to} \quad & u \leq 1, &(C_1)\\
& v \leq 1, &(C_2)\\
& u + v + w \leq 1, &(C_3)\\
& v + w \leq 1, &(C_4)
\end{align*}

Name the above four constraints $C_1, C_2, C_3, C_4$ respectively. \\

Our objective function is $4u + 4v + w$ = $3C_1 + 3C_2 + C_3 \leq 7$. \\

Therefore, the optimal value is 7.

\newpage

\section*{Exercise 3}
Determine whether $(5, 4, 0)$ is the optimal solution to the LP from problem 1 using complementary slackness. \\

\textbf{Solution:} \\

Verify if the point $(5,4,0)$ satisfies all constraints:
\begin{align*}
&2(5) + 4 + 2(0) = 14 \leq 14 \checkmark\\
&5 + 0 = 5 \leq 8 \checkmark\\
&2(5) + 2(4) - 0 = 18 \leq 18 \checkmark
\end{align*}

The point is feasible. The dual LP is:
\begin{align*}
\min_{u, v, w \in \mathbb{R}} \quad & 14u + 8v + 18w \\
\text{subject to} \quad & 2u + v + 2w \geq 1, \\
& u + 2w \geq 1, \\
& 2u + v - w \geq 1, \\
& u, v, w \geq 0
\end{align*}

Now we find $(u^*, v^*, w^*)$ that satisfies complementary slackness with $(5,4,0)$:
\begin{itemize}
    \item $v^* = 0$, since $5 + 0 = 5 < 8$
    \item $2u^* + v^* + 2w^* = 1$, since $5 > 0$
    \item $u^* + 2w^* = 1$, since $4 > 0$
\end{itemize}

Hence, we get $(u^*, v^*, w^*) = (0, 0, 0.5)$. \\

Now we check if $(u^*, v^*, w^*)$ is dual feasible:

\begin{align*}
2(0) + 0 + 2(\frac{1}{2}) &= 1 \geq 1 \checkmark\\
0 + 2(\frac{1}{2}) &= 1 \geq 1 \checkmark\\
2(0) + 0 - \frac{1}{2} &= -\frac{1}{2} \not\geq 1 \times
\end{align*}

Therefore, $(u^*, v^*, w^*)$ is not dual feasible. \\

By complementary slackness, we conclude that $(5,4,0)$ is not the optimal solution.

\newpage

\section*{Exercise 4}
Consider the following LP:
\begin{align*}
\max_{x \in \mathbb{R}^n} \quad & c^{\top} x \\
\text{subject to} \quad & a^{\top} x \leq 1, \\
& x \geq 0
\end{align*}
where $a, c \in \mathbb{R}^n$ and $a, c > 0$.

\begin{enumerate}[(a)]
    \item Write down the dual LP.
    \item Determine the optimal dual solution.
    \item Find a primal solution with the same objective value.
\end{enumerate}

\textbf{Solution:}

Since I registered 3 credits section, I would like to skip this problem.

\newpage

\section*{Exercise 5}
Consider the following LP:
\begin{align*}
\max_{x \in \mathbb{R}^n} \quad & c^{\top} x \\
\text{subject to} \quad & Ax \leq b, \\
& x \geq 0
\end{align*}

Derive conditions on $A$, $b$, and $c$ such that the dual of the above is identical to the LP itself. \\

\textbf{Solution:} \\

The dual LP is:
\begin{align*}
\min_{u \in \mathbb{R}^m} \quad & b^{\top} u \\
\text{subject to} \quad & A^{\top} u \geq c, \\
& u \geq 0
\end{align*}

First for the objective function, both problems must have the same form, like both maximization or both minimization. Without loss of generality, we can convert the dual minimization to maximization: $\max_{u \in \mathbb{R}^m} -b^{\top} u$.
For the objectives to match: $c^{\top} x = -b^{\top} u$ when $x = u$, which gives us $c = -b$. \\

Second for the constraints, both problems must have the same of them: Negate the dual constraint: $-A^{\top}u \leq -c$. For this to match $Ax \leq b$ when $x = u$: $-A^{\top} = A$ and $-c = b$, which gives us: $A^{\top} = -A$ and actually helps us confirm $c = -b$ \\

Therefore, the conditions are: $A^{\top} = -A$ and $c = -b$.

\end{document}