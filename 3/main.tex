\documentclass{article}
\usepackage{graphicx}
\usepackage{amsmath}
\usepackage{array}
\usepackage{fancyhdr}
\usepackage{amssymb}
\usepackage[shortlabels]{enumitem}

\DeclareMathOperator{\R}{\mathbb R}

\pagestyle{fancy}
\fancyhead[L]{Banghao Chi}
\fancyhead[C]{Homework 3}
\fancyhead[R]{28th Feb}

\fancyfoot[C]{\thepage}

\renewcommand{\headrulewidth}{0.5pt}
\renewcommand{\footrulewidth}{0.5pt}

\begin{document}

\section*{Exercise 1}
Consider the following LP:
\begin{align*}
\max_{x, y, z \in \mathbb{R}} \quad & x + y + z \\
\text{subject to} \quad & 2x + y + 2z \leq 14, \\
& x + z \leq 8, \\
& 2x + 2y - z \leq 18, \\
& x, y, z \geq 0
\end{align*}

Write down the dual LP. \\

\textbf{Solution:} \\

\text{The dual LP is:} \\
\begin{align*}
\min_{u, v, w \in \mathbb{R}} \quad & 14u + 8v + 18w \\
\text{subject to} \quad & 2u + v + 2w \geq 1, \\
& u + 2w \geq 1, \\
& 2u + v - w \geq 1, \\
& u, v, w \geq 0
\end{align*}

\newpage

\section*{Exercise 2}
Solve the dual of the following LP (no need to find an optimal vector, just the optimum value):
\begin{align*}
\min_{x \in \mathbb{R}^4} \quad & x_1 + x_2 + x_3 + x_4 \\
\text{subject to} \quad & x_1 + x_3 = 4, \\
& x_2 + x_3 + x_4 = 4, \\
& x_3 + x_4 = 1, \\
& x \geq 0
\end{align*}

\textbf{Solution:}



\newpage

\section*{Exercise 3}
Determine whether $(5, 4, 0)$ is the optimal solution to the LP from problem 1 using complementary slackness. \\

\textbf{Solution:}



\newpage

\section*{Exercise 4}
Consider the following LP:
\begin{align*}
\max_{x \in \mathbb{R}^n} \quad & c^{\top} x \\
\text{subject to} \quad & a^{\top} x \leq 1, \\
& x \geq 0
\end{align*}
where $a, c \in \mathbb{R}^n$ and $a, c > 0$.

\begin{enumerate}[(a)]
    \item Write down the dual LP.
    \item Determine the optimal dual solution.
    \item Find a primal solution with the same objective value.
\end{enumerate}

\textbf{Solution:}



\newpage

\section*{Exercise 5}
Consider the following LP:
\begin{align*}
\max_{x \in \mathbb{R}^n} \quad & c^{\top} x \\
\text{subject to} \quad & Ax \leq b, \\
& x \geq 0
\end{align*}

Derive conditions on $A$, $b$, and $c$ such that the dual of the above is identical to the LP itself. \\

\textbf{Solution:}



\end{document}